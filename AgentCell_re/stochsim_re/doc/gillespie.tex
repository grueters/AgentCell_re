\chapter[Comparison with the Gillespie algorithm]{\stochsim{}: general description and comparison with the Gillespie algorithm}

The computer program \stochsim{} was written by Carl Firth as part of his
PhD work at the University of Cambridge (Morton-Firth, 1998). It was
developed as part of a study of bacterial chemotaxis as a more
realistic way to represent the stochastic features of this signalling
pathway and also as a means to handle the large numbers of individual
reactions encountered (Morton-Firth \& Bray, 1998; Morton-Firth et
al., 1999). The program provides a general purpose biochemical
simulator in which individual molecules or molecular complexes are
represented as individual software objects. Reactions between
molecules occur stochastically, according to probabilities derived
from known rate constants.  An important feature of the program is its
ability to represent multiple post-translational modifications and
conformational states of protein molecules.

\stochsim{} consists of a platform-independent core simulation engine 
encapsulating the algorithm described above and separate graphical user 
interfaces.   

\section{Description of the algorithm}

Each molecule or molecular species is represented as a separate software object
in \stochsim{}, and the simulation also includes dummy molecules, or 
``pseudo-molecule'' used in the simulation of unimolecular reactions. Time is 
quantised into a series of discrete, independent time-slices, the size of 
which is determined by the most rapid reaction in the system. In each 
time-slice, one molecule (not a pseudo-molecule) is selected at random. Then, 
another object, in this case either a molecule or a pseudo-molecule, is 
selected at random. If two molecules are selected, any reaction that occurs 
will be bimolecular, whereas if one molecule and a pseudo-molecule are 
selected, it will be unimolecular. Another random number is then generated and 
used to see if a reaction occurs; the probability of a reaction is retrieved 
from a look-up table and if the probability exceeds the random number, the 
particles do not react. On the other hand, if the probability is less than the 
random number, the particles react, and the system is updated accordingly. The 
next time-slice then begins with another pair of molecules being selected. 

Whenever a molecule in the system can exist in more than one state then the 
program encodes it as a ``multistate molecule'' with a series of binary flags. 
Each flag represents a state or property of the molecule, such as a 
conformational state; ligand binding; phosphorylation, methylation, or other 
covalent modification. The flags specify the instantaneous state of the 
molecule and may modify the reactions it can perform. For instance, a 
multistate molecule may participate in a reaction at an increased rate as a
result of phosphorylation, or fail to react because it is in an inactive 
conformation. The flags themselves can be modified in each time step as a 
result of a reaction, or they can be instantaneously equilibrated according to 
a fixed probability. The latter tactic is used with processes such as ligand 
binding or conformational change that occur several orders of magnitude faster 
than other chemical reactions in the system. 

If, in a particular time step, \stochsim{} selects one or more multistate 
molecules, then it proceeds in the following manner. First any 
rapidly-equilibrated ``fast flags'' on the molecule are assigned to be on or 
off according to a weighted probability. A protein conformation flag, for 
example, can be set to be active or inactive, according to which other flags 
of the molecules are currently on. A ligand binding flag can, if desired, be 
set in a similar fashion, based on the concentration of ligand and the Kd. 

Once the fast flags have been set, then the program inspects the reactions 
available to species A and B. The chemical change associated with each type of 
reaction (binding, phosphotransfer, methylation, etc.) is represented in the 
program together with a "base values" of the reaction rate constants. The 
particular instantiation of the reaction, determined by the current state of 
the flags on A and B, is accessed from an array of values calculated at the 
beginning of the program, when the reaction system is being initialised. 
Values in the array modify the reaction probability according to the particular
set of binary flags. In this manner, the \stochsim{} calculates a set of 
probabilities, corresponding to the reactions available to the particular 
states of molecules A and B, and then uses a random number to select which 
reaction (if any) will be executed in the next step. The reaction will be 
performed, if appropriate, and the relevant slow flag flipped. 

Although it sounds complicated, the above sequence of events within an 
individual iteration takes place very quickly and even a relatively slow 
computer can carry out hundreds of thousands of iterations every second. 
Moreover, the strategy has the advantage of being intuitively simple and close 
to physical reality. For example, it is easy, if required, to label selected
molecules and to follow their changes with time. Lastly, the speed of the 
program depends not on the number of reactions but on the numbers of molecular 
species in the simulation (with a time of execution proportional to N squared).
  
\section{Comparison with the Gillespie algorithm}

The stochastic simulation of biochemical reactions was pioneered by
Gillespie, who developed an elegant and efficient algorithm for this
purpose (Gillespie, 1976; Gillespie, 1977). Gillespie showed in
rigorous fashion, that his algorithm gives the same result, on
average, as conventional kinetic treatments. In ensuing years, the
algorithm has been widely used to analyse biochemical kinetics and,
most recently, to simulate the stochastic events in lambda lysogeny
(McAdams \& Arkin, 1997). In view of its evident success, the question
therefore arises: Why in our work we did not use the Gillespie
algorithm but chose to develop our own formulation?

The Gillespie algorithm makes time steps of variable length, based on
the rate constants and population size of each chemical species. The
probability of one reaction occurring relative to another is obtained
by multiplying the rate constant of each reaction with the numbers of
its substrate molecules. A random number is then used to choose which
reaction will occur, based on relative probabilities, and another
random number determines how long the step will last. The chemical
populations are altered according to the stoichiometry of the reaction
and the process is repeated. Perhaps because the algorithm was
developed at a time when computers were several thousand times slower
than they are today, it makes extremely efficient use of CPU time. At
each iteration it selects the reaction most likely to occur and
chooses a time step that optimises that reaction, so that the
simulation proceeds extremely efficiently.

However, the efficiency of the Gillespie algorithm comes at a cost.
The elegant algorithm that selects which reaction to perform, and what
time interval to take, cannot represent individual molecular events in
the reaction. With regard to the reactions of a typical cell
signalling pathway, for example, it cannot associate physical
quantities with each molecule, nor trace the fate of particular
molecules over a period of time. This means that it is not possible to
extend this algorithm to a more thermodynamically realistic model in
which energies and conformational states are associated with each
molecule.  Similarly, without the ability to associate positional and
velocity information with each particle, the algorithm cannot be
adapted to simulate diffusion, localisation or spatial heterogeneity.

A second deficiency of the Gillespie algorithm (from a cell biological
standpoint) is that it cannot easily handle the reactions of
multi-state molecules. Protein molecules are very frequently modified
in the cell so as to alter their catalytic activity, binding affinity
and so on. Cell signalling pathways, for example, carry information in
the form of chemical changes such as phosphorylation or methylation,
or as conformational state. A multi-protein complex may contain
upwards of twenty sites, each of which can often be modified
independently and each of which can, in principle, influence how the
complex will participate in chemical reactions. With twenty sites, a
complex can exist in a total of 2\textsuperscript{20}, or one million,
unique states, each of which could react in a slightly different way.
If our multi-protein complex interacts with only ten other chemical
species, a detailed model may contain as many as ten million distinct
chemical reactions,  a combinatorial explosion.  Any program in which
the time taken increases in proportion to the number of reactions, as
in a conventional, deterministic model, or in the Gillespie method,
will come to a halt under these conditions.

To summarise, \stochsim{} is likely to be slower than the Gillespie
algorithm in calculating the eventual outcome of a small set of simple
biochemical reactions, especially when the numbers of molecules is
large. However, if the system contains molecules that can exist in
multiple states, then \stochsim{} will not only be faster but also
closer to physical reality. It is easy, if required, to label selected
molecules in this program and to follow their changes with time,
including changes to their detailed post-translational modification
and conformational state. Although the program does not, in its
present form, incorporate spatial information regarding the positions
of molecules, we have found that such modification can be made in a
straightforward manner.
%%% Local Variables: 
%%% mode: latex
%%% TeX-master: "stochsim_manual"
%%% End: 
