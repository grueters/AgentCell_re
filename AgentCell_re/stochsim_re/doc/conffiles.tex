\chapter{The configuration files}\label{conf_files}

The format of \stochsim{} configuration files follows that of Windows
initialisation files. Sections are headed by a title between square
brackets.  Inside each section, one parameter per line is specified in
the form, \emph{parameter = value}.  Comments can be added after a
semicolon.

\begin{verbatim}
[A Section]
parameter1=value         ; A comment
parameter2=value

; A 
; multiline
; comment

[Another Section]
parameterA=value
parameterB=value
\end{verbatim}

\section{The main configuration file (\texttt{STCHSTC.INI})}
This file (\texttt{STCHSTC.INI}) describes the general configuration of a
simulation, independently of the objects and the reactions.

\subsection{[Simulation Parameters]}
\begin{description}
\item[DisplayInterval] The interval between the two displays of variable values, 
in units of seconds (floating point) or simulation
  iterations (integer).  The units are specified by the
  \textbf{TimeUnits} parameter in the \textbf{[Options]} section).

\item[StoreInterval] The interval between two storages of variable values, 
  in units of seconds (floating point) or simulation iterations
  (integer).  The units are specified by the \textbf{TimeUnits}
  parameter in the \textbf{[Options]} section).
  
\item[TimeIncrement] The duration of each simulation iteration, in
  units of seconds (floating point) or simulation iterations
  (integer).  The units are specified by the \textbf{TimeUnits}
  parameter in the \textbf{[Options]} section).
  
\item[SimulationDuration] The length of the simulation, in units of
  seconds (floating point) or simulation iterations (integer).  The
  units are specified by the \textbf{TimeUnits} parameter in the
  \textbf{[Options]} section).
  
\item[ReactionVolume] The total volume of the reaction system, in
  litres (floating point; the exponential form, e.g. ``1.41e-15'', can
  also be used).
  
\item[MaximumNumberComplexes] The maximum number of complexes (not the
  number of complex types!) allowed in the reaction system (integer).
  Note that this number should be large enough to account for the
  total number of complexes that may need to be represented at any
  point in the simulation (not just the number of complexes at the
  beginning of the simulation).
\end{description}

\subsection{[Options]}
\begin{description}
\item[AbortOnResolutionErr] Abort the simulation with an error message
  if a reaction probability is too small for the resolution of the
  random number generator (\mbox{1 = Yes}; \mbox{0 = No}).
  
\item[DisplayAllLevels] Indicates whether or not all variable values
  should be displayed during simulation (1 = Display level of every
  defined complex type; 0 = Use only specified variables).
  
\item[OptimiseTimeIncrement] Optimise time increment for the defined
  reaction system to the maximum value possible without compromising
  the accuracy of simulation (1 = Yes; 0 = No, use time increment
  specified in the \textbf{TimeIncrement} parameter of the
  \textbf{[General]} section).

\item[RandomNumberGenerator] Type of random number generator to use:
  \begin{itemize}
  \item[1] = Dummy generator
  \item[2] = Internal random number generator
  \item[3] = Shuffle random number generator
  \item[4] = Bit string random number generator
  \item[5] = Quick random number generator
  \end{itemize}
  
\item[TimeUnits] The units to be used for parameter values
  representing time (1 = Seconds; 0 = Iterations).
  
\item[CreateDumpFile] Create a dump file of the reaction system (1 =
  Yes; 0 = No).  A dump file stores the state of the entire reaction
  system, including the state of each multistate complex.
\end{description}

\subsection{[File Names]}
\begin{description}
\item[ComplexINIFile] Input file containing details of the components
  and complex types in the system.
       
\item[DumpOut] Output file to which the state of the entire reaction
  system is to be dumped.
  
\item[DynamicValues] Input file containing information about objects
  that change over time.
  
\item[LogFile] Output file in which all messages issued by the
  simulator are to be saved.
  
\item[MessageINIFile] Input file containing all the constant-string
  messages used by the system.
  
\item[ReactionMatrixOut] Output file in which the reaction matrix used
  for this simulation is to be saved.
  
\item[ReactionINIFile] Input file containing reactions and reaction
  constants.
  
\item[SeedsIn] Input file from which the random number seeds are to be
  loaded; the value of this parameter should be left blank if the seeds
  are to be generated from the internal CPU clock.
  
\item[SeedsOut] Output file in which random number seeds are to be
  saved; the value of this parameter should be left blank if seeds
  need not be saved.

\item[VariablesOut] Output file in which variable values are saved.
\end{description}

\section{The complex configuration file (\texttt{COMPLEX.INI})}

This file (\texttt{COMPLEX.INI}) defines the complexes which can take part in
the reactions.  The components of the complexes must be defined first,
then the complexes can be defined. Finally, the initial concentrations
of the complexes must be set. For multistate complexes, additional
information must be defined in a specific INI file for each multistate
complex type.

\subsection{[General]}
\begin{description}
\item[Symbols] A comma-separated list of symbols to be used to denote
components and complex types.  They are not reacting entities, but are the
constituents of the complexes which react.
       
\item[NumDisplayVariables] Number of variables (excluding those for
  multistate complex types) which will be displayed during the
  simulation.
\end{description}

\subsection{[Component \emph{symbol}]}
This section specifies information about the components, i.e. the
elementary blocks of the simulation. Each component must have a
section headed [Component \emph{symbol}], where \emph{symbol} is a
unique string of less than \MAXCOMPONENTSYMBOLLENGTH{} characters. The
maximum number of components allowed is \MAXCOMPONENTS{}.

\begin{description}
\item[Name] The name of the component (must be less than
\MAXCOMPONENTNAMELENGTH{} characters).

\item[Description] The description of the component.  This parameter
is not mandatory, since it is not used by the \stochsim{} simulator
itself.
\end{description}

\subsection{[Complex Types]}
This section specifies information about the complexes (excluding
multistate complexes).

\begin{description}
% \item[NumRapidEqm] Obsolete. Used to be used for Fuzzy complex types
% (probably before MS-Complexes were implemented).
  
\item[NumberOfSets] This number indicates how many lines of complex
  types follow;
  
\item[Set\emph{x}] Each line defining a complex type must start with
  Set\emph{x}, where \emph{x} is the line number. Note that the
  numbers have to be consecutive, and must start with \textbf{1}.  You
  can put one complex symbol per line, or a list of complex symbols,
  joined with commas (\textbf{strongly discouraged}. This feature is
  not handled by the Tk interface).  The maximum number of
  complex types that can be defined in a simulation is
  \MAXCOMPLEXTYPES{}. The symbol of the complex is a concatenated
  string consisting of the symbols representing components of the
  complex type.  The maximum number of components per complex is
  \MAXCOMPONENTSINCOMPLEX{}.
  
%% suppressed for the moment (parsing of the complex symbol)
%\item[Comp\emph{x}] These lines are not read by the \stochsim{} simulator
%  itself, but by the helper programs (e.g. Tk based interface).
%  Each line contains a list of component symbols that form one complex
%  type.  Note the order is important here! (e.g. A,B,C,D forms a
%  different complex from B,C,D,A)

\item[Description\emph{x}] The description of the complex.  This parameter
is not mandatory, since it is not used by the \stochsim{} simulator
itself.
\end{description}

\subsection{[Initial Levels]}
This section specifies initial levels of each complex. Levels are
expressed in number of molecules, \emph{not a concentration}, so
values must be given in \textbf{integers}.  Note that for multistate
complex types, only the total level is specified here (the level of
each state is specified in the multistate configuration file).  One
line of the following format per complex must be defined in this
section: \\[\baselineskip] \emph{Symbol}=\emph{Level}
\\[\baselineskip] 
where \emph{Symbol} is the symbol for a complex type, and \emph{Level}
is its initial level.

\subsection{[Special Complex Types]}
This section specifies what special complex types are being used
(currently, only multistate complex types fall under this category).

\begin{description}
\item[SpecialTypes] Define any special complex types to be used in the
  simulation. Currently, the value of this parameter can be either
  ``Multistate\_Complex'' or empty.
  
\item[Multistate\_Complex] A comma-separated list of symbols, one for
  each multistate complex type in the reaction system.

\item[\emph{S}INIFile] The file containing the specifications of the multistate
       complex \emph{S}. Define one line of this format per multistate
       complex being used.
\end{description}

\subsection{[Display Variable \emph{X}]}
If the parameter \textbf{DisplayAllLevels} (in the \textbf{[Options]}
section of \texttt{STCHSTC.INI}) has been set to 0, you must specify the
variables to be displayed here. Each display variable must have a
section headed [Display Variable \emph{X}], where \emph{X} is a unique
numerical identifier for each display variable for this complex type,
starting with \textbf{1}.

\begin{description}
\item[Name] Name of the variable.
\item[Types] A comma-separated list of the complex type(s) this
  variable represents.
\end{description}


\section{The configuration of reactions (\texttt{REACTION.INI})}
\label{reaction_ini}
This file (\texttt{REACTION.INI}) contains the description of the reactions
which can take place during the simulation. Each reaction is
considered to be a reversible reaction with separate forward and
reverse rate constants. For an irreversible reaction, set the reverse
rate to zero. The rate constants are expressed in standard units,
according to the order of the reaction (s\textsuperscript{-1} is used
for unimolecular reactions, and
M\textsuperscript{-1}s\textsuperscript{-1} for bimolecular reactions
).

\subsection{[General]}
This section contains only one parameter.

\begin{description}
\item[NumberOfReactions] The number of reactions in the simulation
(integer).
\end{description}

\subsection{[Reaction \emph{X}]}
The section for each reaction should start with a title [Reaction
\emph{X}] where \emph{X} is the reaction number.
\begin{description}
\item[Description] An explicit description of the reaction. This
  parameter is not mandatory, since it is not used by the \stochsim{}
  simulator itself.

\item[Substrates] A comma-separated list of substrates.

\item[Products] A comma-separated list of products.
  
\item[kf] Forward rate constant (floating point). If a dynamic value
  is being used, use the form, \emph{X}@\emph{Code} (see section
  \ref{r_dv_notes}).
  
\item[kr] Reverse rate constant (floating point). If a dynamic value
  is being used, use the form, \emph{X}@\emph{Code} (see section
  \ref{r_dv_notes}).
\end{description}

\subsection{Notes on using dynamic values in this file (\texttt{REACTION.INI})}
\label{r_dv_notes}
If the rate constants can change over time, dynamic values must be
used.  To specify a reaction rate using a dynamic value, use the
following form:\\[\baselineskip]
\emph{X}@\emph{Code} where \emph{X} is the maximum rate the reaction
can achieve and \emph{Code} is the symbolic code identifying this rate
constant in the file controlling the value dynamically changing over
time. eg 15@Reaction1.  The actual behaviour of dynamic values are
configured in a separate configuration file (DYNAMIC.INI).


\section{The configuration of dynamic values (\texttt{DYNAMIC.INI})}\label{dynamic_ini}
This file (\texttt{DYNAMIC.INI}) contains the configuration of the dynamic
values, i.e.  the details of the objects which change over time. There
are three types of objects which can change over time:

\begin{itemize}
\item Reaction rates (see section \ref{reaction_ini})
\item Rapid equilibrium probabilities for multistate complexes
\item Reaction rates for multistate complexes (see section
  \ref{ms_ini})
\end{itemize}

A number of time points are listed in the \textbf{[General]} section
of this file. The value of all dynamic values must be assigned for
each time point.  All times are either in seconds or iterations
depending on what was specified for units in the main configuration
file.
% When dynamic values are used in the definition of objects in other INI
% files, a code identifying the listing in this file.

\subsection{[General]}
\begin{description}
\item[NumberOfSets] The number of time points in the simulation at
  which the dynamic values change.

\item[Set\emph{X}] Each set contains a list of time-points at which
  dynamic values change.
\end{description}

\subsection{[Time \emph{T}]}
Each \textbf{[Time \emph{T}]} section must contain a line for each
dynamic value, with its value at time \emph{T}. Each line of this
section has the form:
\\[\baselineskip] 
\emph{Code}=\emph{Value}
\\[\baselineskip]
where \emph{Code} is the code of the dynamic value and
\emph{Value} is its actual value at time \emph{T}.  \emph{Value} may
be left blank for dynamic values that do not change at time \emph{T}.

\section{Configuration of the multistate complexes (\texttt{MS\_\emph{X}.INI})}
\label{ms_ini}

If multistate complexes are to be used in a simulation, a separate
configuration file must be defined for each multistate complex type.
These files can take any name, as long as it matches that specified in
the complex configuration file (\texttt{COMPLEX.INI}).  However, we recommend
using file names of the following form, which both the Perl/Tk and
MFC GUIs recognise as multistate configuration files:
``\texttt{MS\_\emph{X}.INI}'', where \emph{X} is a unique numerical identifier
for each multistate complex type in the system, starting with
\textbf{1}.\par

The multistate configuration file contains all the additional
information about multistate complexes (MS-Complexes), multistate
reactions (MS-Reactions) and multistate rapid equilibria
(MS-RapidEqm's) needed by \stochsim{}, which has not already been given
in the other four configuration files (\texttt{STCHSTC.INI}, \texttt{COMPLEX.INI},
\texttt{REACTION.INI} and \texttt{DYNAMIC.INI}).  For example, the initial level (or
more precisely, the initial \emph{total} level) of an MS-Complex type
is already given in the complex configuration file (\texttt{COMPLEX.INI}), but
the initial levels for each state of the multistate complex must be
defined here (in MS\_\emph{X}.INI).  Similarly, all simple
unimolecular and bimolecular reaction rates are given in the reaction
configuration file (\texttt{REACTION.INI}), but information specific to
MS-Reactions (i.e. all reactions involving MS-Complexes), such as how
the different states of an MS-Complex will affect reaction rates, and
the effect that an MS-Reaction has on the state flags of product
MS-Complexes, must be specified here.

\subsection{How multistate molecules work}
In \stochsim{}, the state of an MS-Complex can affect two things:

\begin{description}
\item[Reaction Rates] \stochsim{} reactions and their rates are defined
  in the reaction configuration file (\texttt{REACTION.INI}).  However,
  reactions involving MS-Complexes (we will call these reactions
  MS-Reactions) are treated differently by the simulator, and
  therefore require some additional information before the simulation
  can begin.  The difference between MS-Reactions and normal \stochsim{}
  reactions are:
  \begin{description}
  \item[How reaction rates are computed] Normal \stochsim{} reaction
    rates are determined only by the information in the reaction
    configuration file (\texttt{REACTION.INI}).  The rate of MS-Reactions can
    be affected by the state(s) of participating MS-Complex(es).
    
  \item[How the system is updated afterwards] After normal \stochsim{}
    reactions occur, the reaction system must be updated by adding
    product complexes, or removing substrate complexes as necessary.
    When MS-Reactions occur, it may also be necessary to update the
    state of individual MS-Complexes, i.e. certain state flags of
    participating MS-Complexes may have to be turned on or off.

  \end{description}
  
\item[Rapid Equilibrium Probabilities] MS-RapidEqm's are a facility of
  MS-Complexes which helps to deal with stiffness in a reaction
  system.  If certain state flags are controlled by very fast
  processes (such as conformational changes of a protein), defining
  these state changes as MS-Reactions would force the time-increment
  of the simulation to be made very small.  This can cause a very
  large increase in simulation time.\par
  
  Instead, \stochsim{} allows such flags to be controlled by
  MS-RapidEqm's.  When a complex is chosen for reaction, \stochsim{} will
  first ``equilibrate'' any rapid equilibria defined for the complex.
  Here, to ``equilibrate'' a binary flag simply means to decide
  whether it is in the on or off state, according to a predefined
  \emph{rapid equilibrium probability}.\par
  
  Now, a rapid equilibrium probability can itself be affected by the
  state of other flags within the same MS-Complex.  For example, the
  conformational equilibrium of a protein may be affected by the
  binding of a ligand molecule, or covalent modification at a certain
  site.  By using MS-RapidEqm's effectively, such protein complexes in
  signalling pathways can be modelled with relative ease.

\end{description}


\subsection{[General]}
\begin{description}
\item[StateFlags] A comma-separated list of state flag names (the
  maximum length for the name of each flag is \MAXMSFLAGLENGTH{}
  characters).
  
\item[NumRapidEqm] The number of flags controlled by rapid equilibria
  (integer).
  
\item[Reactions] A comma-separated list of all reactions involving
  this MS-Complex. Reaction names are given in the form ``\emph{XD}'',
  where \emph{X} is the reaction number and \emph{D} is the direction
  (`F' for forward and `R' for reverse).  For example, ``4F'' for the
  forward sub-reaction of reaction 4, as it appears in the reaction
  configuration file (\texttt{REACTION.INI}).
  
\item[NumDisplayVariables] The number of multistate variables
  (MS-Variables) which will be displayed (see section
  \ref{ms_variables}).
\end{description}

\subsection{[Initial States]}
This section defines the initial states of the MS-Complex.
\begin{description}
\item[Wildcards] A comma-separated list of all bit strings below which
  contain one or more of the wildcard character, '?'. Wildcards
  represent \emph{both} 0 and 1.

\item[\emph{BitString}] As many lines as necessary of the
following form can be written here to define the initial level of each
state of the MS-Complex:\\[\baselineskip]
\emph{BitString}=\emph{Value}
\\[\baselineskip]
where \emph{BitString} is a bit string that specifies a particular
state, and \emph{Value} is the number of MS-Complexes in that state.
Note that wildcard strings can be used in place of bit strings here to
specify initial levels of multiple states in a single line.
\end{description}

\subsection{[Rapid Equilibrium \emph{X}]}
Each MS-RapidEqm must have a section headed [Rapid Equilibrium
\emph{X}] where \emph{X} is a unique numerical identifier for each
MS-RapidEqm of this MS-Complex, starting with \textbf{1}.

\begin{description}
\item[Flag] The name of the state flag which is controlled by the
  rapid equilibrium. ``State'' is also accepted for backward compatibility but
  is deprecated.
  
\item[Wildcards] Comma-separated list of all bit strings below which
  contain one or more of the wildcard character, '?'. Wildcards
  represent \emph{both} 0 and 1.
       
\item[\emph{BitString}] As many lines as necessary of the following
  form can be written here to define the MS-RapidEqm probabilities
  (i.e. the probability that the flag controlled by this MS-Rapid Eqm
  will be on)
  associated with particular states of this MS-Complex:\\[\baselineskip]
  \emph{BitString}=\emph{Value}
  \\[\baselineskip]
  where \emph{BitString} is a bit string that specifies a particular
  state (note that wildcard strings can be used in place of bit
  strings in these lines to specify initial levels of multiple states
  in a single line).  \emph{Value} is the probability expression,
  which can take one of two forms:\\[\baselineskip]
  \emph{P\down{on}}\\[\baselineskip]
  where \emph{P\down{on}} is simply the probability of the flag
  being on (floating point), or\\[\baselineskip]
  \emph{T\down{off}},\emph{T\down{on}}\\[\baselineskip]
  where \emph{T\down{off}}:\emph{T\down{on}} is the ratio, (time
  spent in the off state):(time spent in the on state).  For
  example, the following two lines are equivalent:\\[\baselineskip]
  ????=0.75\\[\baselineskip]
  ????=1.0,3.0\\[\baselineskip]
  So you can use either a single floating point value, or a ratio of
  two floating point values.

\end{description}

\subsection{[Reaction \emph{XD}]}
Each reaction involving a multistate complex (an MS-Reaction) must
have a [Reaction \emph{XD}] section where \emph{X} is the reaction
number and \emph{D} is direction ('F' or 'R').  MS-Reactions are
computed by using \emph{relative rates} specific to certain states of
an MS-Complex.  Relative rates are simply factors by which the maximum
rate of reaction (defined in \texttt{REACTION.INI}) are multiplied to obtain a
final reaction probability.  The effect of an MS-Reaction on the
product MS-Complex must also be defined here.
\begin{description}
\item[Effect] A comma-separated list of the effects that this
  MS-Reaction has on the state flags of this MS-Complex.  A '+'
  character indicates that a flag is set (0 => 1); a '-' indicates
  that a flag is cleared (1 => 0).
  
\item[Wildcards] Comma-separated list of all bit strings below which
  contain one or more of the wildcard character, '?'. Wildcards represent \emph{both} 0 and 1.
\item[\emph{BitString}] As many lines as necessary of the following
  form can be written here to define the relative rates of this
  MS-Reaction associated to particular states of this
  MS-Complex:\\[\baselineskip]
  \emph{BitString}=\emph{Value}
  \\[\baselineskip]
  where \emph{BitString} is a bit string that specifies a particular
  state, and \emph{Value} is the relative rate associated with that
  state.  Note that wildcard strings can be used in place of bit
  strings here to specify relative rates for multiple states in a
  single line.

\end{description}

\subsection{[Display Variable \emph{X}]}\label{ms_variables}
If the parameter \textbf{DisplayAllLevels} (in the \textbf{[Options]}
section of \texttt{STCHSTC.INI}) has been set to 0, you must specify the
multistate variables (MS-Variables) to display here.  Each MS-Variable
must have a section headed [Display Variable \emph{X}], where \emph{X}
is a distinct number assigned to each variable.  Note that these
numbers have to be consecutive, and must start with \textbf{1}.

\begin{description}
\item[Name] Name of the variable.
\item[States] A comma-separated list of the states of this MS-Complex
  that this variable represents.
\end{description}

\subsection{Notes on using dynamic values in this file (\texttt{MS\_\emph{X}.INI})}
\begin{itemize}
\item If the relative rates associated with an MS-Reaction can change
  over time, then they should be specified as (for each bit
  string or wildcard string):\\[\baselineskip]
  \emph{X}@\emph{Code} where \emph{X} is the maximum relative rate
  associated with this dynamic value (floating point), and \emph{Code}
  is the symbolic code identifying this rate constant in \texttt{DYNAMIC.INI}
  (e.g.  0.8@Reaction1).\par
  
\item If an MS-RapidEqm probability can change over time, then it can
  be specified in one of two ways (for each bit string or
  wildcard string): \\[\baselineskip]
  \emph{X},\emph{Y}@\emph{Code} where \emph{X} is the probability that
  the state flag is off and \emph{Y} is the probability that the state
  flag is on and \emph{Code} is a symbolic code identifying this value
  in the dynamic values configuration file (\texttt{DYNAMIC.INI}).  e.g.
  1,1@RE1.\par
  Alternatively,\\[\baselineskip]
  \emph{X}@\emph{Code} where \emph{X} is the probability that the
  state flag is on and \emph{Code} is a symbolic code identifying this
  value in the dynamic values configuration file (\texttt{DYNAMIC.INI}).  e.g.
  0.5@RE1;
\end{itemize}

%%% Local Variables: 
%%% mode: latex
%%% TeX-master: "stochsim_manual"
%%% End: 
