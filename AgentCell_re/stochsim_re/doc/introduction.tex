\chapter{Introduction}
\stochsim{} is a discrete, stochastic simulator. It employs a simple,
novel algorithm in which enzymes and other protein molecules are
represented as individual software objects interacting according to
probabilities derived from concentrations and rate constants.
Formally, it is a mesoscopic simulator, meaning that it stores an
internal representation of every molecule in the system as a unique
object, but does not simulate diffusion.  The program was written in
standard C++, according to the ANSI current working
paper\footnote{This paragraph was written in 1998}.

\section{Overview}
When a simulation is executed, the reaction system is constructed by
creating all the necessary objects in turn. First the event manager is
created, which is responsible for changing the system during the
simulation; for instance, the user may wish to change the
concentration of signalling molecules half way through the simulation.
Then the random number generator is created, based on an algorithm
which breaks up any sequential patterns using a shuffle table. Objects
representing each type of molecular species in the system are
initialised and then large numbers of objects each representing an
individual molecule are created. It is possible to create molecules
which have specific states, called multistate molecules; these react
according to the state they are in and are usually used to reflect
covalent modification, such as protein phosphorylation. A number of
dummy, or pseudo-molecules, are also created at this time, which are
used in the simulation of unimolecular reactions: if a molecule reacts
with a pseudo-molecule, the former may undergo a unimolecular
reaction. Next, a look-up table is constructed which defines all the
possible ways in which molecules can react in the system. For every
bimolecular reaction, the row is selected according to the first
reactant and the column by the second reactant; the corresponding
entry in the look-up table then gives the probability that these two
reactants will react and what the products will be. Finally, objects
are constructed that represent variables being output to the screen
and saved to file. Each variable is responsible for recalculating its
current value as necessary.

\section{The \stochsim{} algorithm}\label{algorithm}
Execution follows a very simple algorithm. Time is quantised into a
series of discrete, independent time-slices. In each time-slice, one
molecule (not a pseudo-molecule) is selected at random. Then, another
object, in this case either a molecule or a pseudo-molecule, is
selected at random. If two molecules were selected, any reaction that
occurs will be bimolecular; if one molecule and a pseudo-molecule were
selected, it will be unimolecular. Another random number is then
generated and used to see if a reaction occurs. The probability of a
reaction is retrieved from the look-up table: if the probability
exceeds the random number, the particles do not react; if the
probability is less than the random number, the particles react, and
the system is updated accordingly.  The next time-slice then begins
with another pair of molecules being selected.

The probabilities that a reaction will occur after the molecules have
been selected are calculated as:
\begin{itemize}
\item If a first molecule A is selected and the second is a
  pseudo-molecule, the probability that A will undergo a
  unimolecular reaction is:
 
 \begin{displaymath}
    P = \frac{k_1\times{}n\times{}(n+n_0)\times\Delta{}t}{n_0}
  \end{displaymath}

\item If two molecules are selected, the 
probability that they react together is:

  \begin{displaymath}
    P = \frac{k_2\times{}n\times{}(n+n_0)\times\Delta{}t}{2\times{}N_A\times{}V}
  \end{displaymath}
\end{itemize}

Where:

\begin{tabular}{lll}
$n$         & = & number of molecules in the system\\
$n_0$       & = & number of pseudo-molecules in the system\\
$k_1$       & = & unimolecular rate constant ($s^{-1}$)\\
$k_2$       & = & bimolecular rate constant ($M^{-1}s^{-1}$)\\
$\Delta{}t$ & = & time-slice duration ($s$)\\
$N_A$       & = & Avogadro constant\\
$V$         & = & volume of the system ($l$)
\end{tabular}


The number of pseudo-molecules is calculated to minimise the stiffness
between the unimolecular and bimolecular reactions. For this we
require that the probability of the fastest unimolecular reaction is
as close to the probability of the fastest bimolecular reaction as
possible. Hence, by equating the two probabilities:

\begin{displaymath}
  n_0=[2\times{}N_A\times{}V\times{}\frac{k_{1,max}}{k_{2,max}}]
\end{displaymath}

Where $[x]$ represents the non-zero positive integer nearest to $x$.

\section{Multistate molecules}\label{intro_multistates}

Besides its discrete, stochastic algorithm, another feature that makes
\stochsim{} unique is its ability encapsulate \emph{internal states} of
molecules within each instance of a molecule object.  The activity of
many enzymes and signalling proteins in living cells are controlled by
numerous factors such as covalent modification, the binding of ligand
or other subunits, and conformational changes of the protein.  It is
possible to model such internal states of a single reacting molecule
as separate molecular species, and this is in fact how most simulators
handle this problem.  However, \stochsim{} provides an alternative
method, which takes advantage of the way molecules are represented as
individual software objects.\par

Molecules that possess many internal states can be modelled as a
special type of molecule, called \emph{multistate} molecules.
Multistate molecules have a set of binary flags (flags which are
either 'on' or 'off'), which can be used to represent the state of the
molecule.  For example, a single flag could be used to express whether
or not an external ligand is bound to a transmembrane receptor.
Another flag could be used to distinguish between two functionally
distinct conformations of the same receptor.  Another could be used to
encode the state of a phosphorylation site on its cytoplasmic domain,
and so on.  The combination of the states of all flags, therefore,
defines the state of a multistate molecule, and this is readily
expressed as bit strings (strings of 0's and 1's), e.g.  ``0010'' when
there is four flags, and only the third one is on).  The state of each
multistate molecule can change over the course of a simulation,
either as a consequence of explicitly computed reactions (following
the algorithm described in section \ref{algorithm}), or by means of
\emph{rapid equilibria}, which are a special feature of multistate
molecules that are provided to deal with stiffness in the reaction
system.

% For explicitly computed reactions involving
% multistate molecules, the maximum rate of each reaction is stored as
% usual as probabilities in the look-up table (see section
% \ref{algorithm}).  An additional table for each multistate molecule is
% constructed to store the \emph{relative probabilities} associated with
% each state, which are factors gives the final reaction probability
% when multiplied with the maximum probability.


% rate is assigned to each possible state of the multistate molecule,
% and is multiplied with the rate


%feature of multistate molecules which allow certain flags which are
%controlled by very fast processes (such as conformational changes of a
%protein) to be defined.

In short, multistate molecules provide a conceptually simpler
alternative to defining the large number of separate molecular species
and associated reactions that would otherwise be necessary.  With
multistate molecules, the user can simply define how each state of a
multistate molecule will affect a reaction rate in relative terms.  An
additional advantage of multistate molecules is that its use can
contribute to computational efficiency when the number of possible
internal states are very large.  This is because \stochsim{} does not
need to sift through a large number of possible reactions at each
simulation interval, and only has to apply the relative effect of the
internal state in the algorithm described above.

%%% Local Variables: 
%%% mode: latex
%%% TeX-master: "stochsim_manual"
%%% End: 


